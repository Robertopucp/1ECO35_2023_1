\begin{table}[]
\caption{Treatment Effects on Other High-Earning Majors}
{
\def\sym#1{\ifmmode^{#1}\else\(^{#1}\)\fi}
\begin{tabular}{@{\extracolsep{2pt}}l*{4}{c}@{}}
\hline\hline


 & Major social sciences & Major arts & Major communication & Major humanities \\
\hline
Treatment class � 2016 & -0.006 & -0.031 & -0.012 & -0.141\sym{***} \\
 & (0.049) & (0.037) & (0.038) & (0.045) \\
Year 2016 & -0.023 & 0.018 & -0.008 & 0.067\sym{**} \\
 & (0.033) & (0.025) & (0.026) & (0.030) \\
Treatment class (in 2015) & -0.037 & 0.035 & -0.002 & 0.083\sym{***} \\
 & (0.034) & (0.026) & (0.026) & (0.031) \\
Constant & 0.624\sym{***} & 0.050 & 0.146\sym{*} & 0.140 \\
 & (0.113) & (0.085) & (0.087) & (0.102) \\
Controls & Yes & Yes & Yes & Yes \\

\hline
Observations & 627 & 627 & 627 & 627 \\
\hline\hline
\multicolumn{5}{l}{\footnotesize Notes: LPM regressions. We report wild bootstrap cluster p-values in parentheses and wild bootstrap cluster 95 per-        cent confidence intervals in square brackets, generated using boottest command in Stata 14 (Roodman et al. 2019)        for standard errors clustered at the class level (12 clusters). Dependent variables: dummy equal to one if the stu-        dent majored in (i) social science (other than economics), (ii) arts, (iii) communication studies, or (iv) humanities.}\vspace{-.25em} \\
\multicolumn{5}{l}{\footnotesize * Significantly different from zero at 90 percent confidence.}\vspace{-.25em} \\
\multicolumn{5}{l}{\footnotesize ** Significantly different from zero at 95 percent confidence.}\vspace{-.25em} \\
\multicolumn{5}{l}{\footnotesize *** Significantly different from zero at 99 percent confidence.}
\end{tabular}
}
\end{table}